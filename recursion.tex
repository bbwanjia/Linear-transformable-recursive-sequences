\documentclass{beamer}
\usetheme{Boadilla}
\usepackage{amsmath}
\usepackage{amssymb}
\theoremstyle{definition}
\newtheorem*{acknowledgement}{Acknowledgement}
\theoremstyle{remark}
\newtheorem*{remark}{Remark}
\title{Solving Recursive Sequences}
\subtitle{the Homogeneous Linear case}
\author{Liu Jingyu}
\date{\today}
\AtBeginSection[]{
  \begin{frame}
    \frametitle{Table of Contents}
    \tableofcontents[currentsection]
  \end{frame}
  }


\begin{document}
\frame{\titlepage}
\section{Introduction}
\begin{frame}
  \frametitle{Fibanacci Sequence}
  The Fibonacci sequence is defined as $a_0 = 0$, $a_1 = 1$ and with 
  recursive formula $a_n = a_{n-1} + a_{n-2}$ for all $n=2,3,4\dots$ 
  \\
  It is well known that, for the Fibonacci sequence there 
  exists a closed-form general formula as below. 
  \\
  \begin{block}{Binet's Formula}
     The general formula of Fibonacci sequence is
     \[
       a_n = \frac{1}{\sqrt{5}} \left[\left(\frac{1+\sqrt{5}}{2}\right)^n
       -\left(\frac{1-\sqrt{5}}{2}\right)^n\right].
     \]
  \end{block}
\end{frame}
\begin{frame}
  \frametitle{First Insight}
  Can we find a closed-form general formula to any \emph{homogeneous linear} 
  recursive sequence, probably involving more backward terms and some 
  parameters on these terms?
  \begin{block}{Problem}
     Can we find a general formula to the following recursion such that 
     \[
       a_n = \sum_{r=1}^t m_r a_{n-r}
     \]
      for all $n \in \mathbb{N} \geq t$ with $a_0 \dots a_{t-1}$ given?
  \end{block}
  Let we start with a simpler case where $t = 2$. 
\end{frame}
\section{Case of $t = 2$}
\begin{frame}
  \frametitle{Solving the special case where $t = 2$}
  \begin{block}{Solution}
    Consider the recursive formula $a_n = m_1 a_{n-1} + m_2 a_{n-2}$. We want 
    to have \[
      a_n - \alpha a_{n-1} = \beta (a_{n-1} - \alpha a_{n-2})
    \]
    so that we can regard $\{a_n - \alpha a_{n-1}\}$ as a 
    geometric sequence. 
    By comparing the coefficients,
    \[
      \begin{cases}
         \alpha + \beta = m_1 \\
         -\alpha\beta = m_2
      \end{cases}
    \]
    So, $\alpha$ and $\beta$ are the roots of the equation \[
      x^2 - m_1 x -m_2 = 0. 
    \] 
  \end{block}
\end{frame}
\begin{frame}
  \begin{block}{Continued}
    Assuming the roots are distinct, we notice that 
    the roots of the equation $x^2 - m_1 x -m_2 = 0$ namely $\alpha$ 
    and $\beta$ are symmetric, so we can write two equations:
    \[
      \begin{cases}
        a_n - \alpha a_{n-1} = \beta (a_{n-1} - \alpha a_{n-2}) \\
        a_n - \beta a_{n-1} = \alpha (a_{n-1} - \beta a_{n-2})
      \end{cases}
    \]
    By considering the initial values of these two 
    corresponding geometric sequences constants $A_1$ and $A_2$, applying
    the general formula for geometric sequences gives \[
      \begin{cases}
        a_n - \alpha a_{n-1} = A_1 \beta^n \\
        a_n - \beta a_{n-1} = A_2 \alpha^n 
      \end{cases}
    \]
    Simplifying give our final formula \[
      a_n = B_1 \alpha^n + B_2 \beta^n
    \]
    where $B_1$ and $B_2$ are constants to be found.
  \end{block}
\end{frame}
\begin{frame}
  \frametitle{An Example}
  \begin{examples}[Fibonacci]
    From the recursive formula $a_n = a_{n-1} + a_{n-2}$ we first 
    form an equation \[
      x^2 - x - 1 = 0.
    \]
    The roots of this equation are $\frac{1 \pm \sqrt{5}}{2}$, 
    so the general formula must be in the form \[
      a_n = A_1 \left(\frac{1+\sqrt{5}}{2}\right)^n
      + A_2 \left(\frac{1-\sqrt{5}}{2}\right)^n.
    \]
    Substituting $a_0 = 0$ and $a_1 = 1$ and solving 
    simultaneous equations directly give 
    \emph{Binet's Formula}.
   \end{examples}
\end{frame}
\section{The More General Case}
\begin{frame} 
  \begin{block}{Conjecture}
     For a recursive sequence with the recursive formula \[
      a_n = \sum_{r=1}^t m_r a_{n-r},
     \]
     we have the general formula \[
       a_n = \sum_{r=1}^{t} C_r \lambda_r^n 
     \]
     where $C_1\dots C_t$ are constants and $\lambda_1\dots \lambda_r$
     are distinct roots of the equation \[
       x^t - \sum_{r=1}^t m_r x^{t-r} = 0.
     \] (All roots are distinct assumed)
  \end{block}
\end{frame}
\begin{frame}
  \frametitle{Concepts}
  We will mainly borrow a concept called \emph{generating function}. 
  \begin{definition}[Generating function]
    For any sequence $a_0, a_1, a_2 \dots$, the \emph{generating function}
    of this sequence is a polynomial defined by \[
      G(x) = a_0 + a_1 x + a_2 x^2 + \dots = \sum a_r x^r.
    \]
    Here $x$ is an arbitrary variable. 
    The function can be finite or infinite.
  \end{definition}
\end{frame}
\begin{frame}
  \frametitle{Concepts}
  \begin{definition}[Characteristic equation]
    For a linear recursive formula \[
       a_n = \sum_{r=1}^t m_r a_{n-r}
    \] its \emph{characteristic equation} is \[
      x^t - \sum_{r=1}^t m_r x^{t-r} = 0.
    \]
  \end{definition}
  One can see that this definition is just an 
  abbreviation of the equation we mentioned 
  several times before. In our conjecture 
  $\lambda_1 \dots \lambda_r$ are distinct
  roots of the characteristic equation.
\end{frame}
\begin{frame}
  \frametitle{Key Idea}
  Our goal is to solve the general formula of the 
  coefficients of the generating function $G(x)$.
  In our case, the generating function is infinite, 
  so it is not easy to directly solve for the general formula. 
  \\
  If we could find a finite function, called $R(x)$, such 
  that \[
    G(x)\cdot R(x) = f(x)
  \] where $f(x)$ is a finite polynomial, then the coefficients 
  of $G(x)$ \textit{could} be solved using Taylor expansion. 
 \\
  We claim explicitly there exists such $R(x)$.
\end{frame}
\begin{frame}
  \begin{block}{Claim}
    Let
   \[
     R(x) = 1 - \sum_{r=1}^t m_r x^r,
   \]
    then $f(x)$ is a finite polynomial with degree $<t$.
  \end{block}
  \begin{proof}
    For all $\varepsilon \in \mathbb{N}\geq t$ we have 
    the term $x^\varepsilon$ in $f(x)$ 
    calculated as \[
         1\cdot a_\varepsilon x^\varepsilon 
         - \sum_{r=1}^t (m_r x^r \cdot a_{\varepsilon-r}x^{\varepsilon-r})
         = \left(a_\varepsilon-\sum_{r=1}^t m_r a_{\varepsilon-r}\right)x^\varepsilon \]
    which is zero according to the recursive formula.
  \end{proof} 
\end{frame}
\begin{frame}
  \frametitle{To a Final Result}
  Hence we have found that, the generating function 
  of the recursive sequence can be represented by
  a quotient of two finite polynomials, i.e. \[
    G(x) = \frac{f(x)}{R(x)} = \frac{f(x)}{1 - \sum_{r=1}^t m_r x^r}. \]
  \begin{examples}[Fibonacci]
     The generating function of Fibonacci sequence
     is \[
       G(x) = 0 + x + x^2 +\dots = \frac{x}{1-x-x^2}.
     \]
  \end{examples}
\end{frame}
\begin{frame}
  \frametitle{Final Result}
  By decomposing it into partial fractions and then 
  implementing Taylor(binomial) expansion, one can find
  the form of the general formula of a homogeneous linear 
  recursive sequence. Simplifying process is omitted here. 
  \begin{alertblock}{Alert}
    When using Taylor(binomial) expansion, it is necessary 
    to consider whether the series is convergent. What is the 
    range of values of $x$ that the expansion is guaranteed to 
    be convergent? This is left as an exercise.
  \end{alertblock}
  Finally we have found that our conjecture is true. 
  The result is shown again on the next slide.
\end{frame}
\begin{frame}
  \begin{theorem}[General formula of homogeneous linear recursion]
     For a recursive sequence with the recursive formula \[
      a_n = \sum_{r=1}^t m_r a_{n-r},
     \]
     we have the general formula \[
       a_n = \sum_{r=1}^{t} C_r \lambda_r^n 
     \]
     where $C_1\dots C_t$ are constants and $\lambda_1\dots \lambda_r$
     are distinct roots of its 
     characteristic equation. It is assumed that
     all roots are distinct. 
 
 \end{theorem}
\end{frame}
\section{Further Thinking}
\begin{frame}
  \frametitle{Exercise}
  \begin{block}{Exercise}
    Find the range of values of $x$ that the
    Taylor(binomial) expansion in our method
    is guaranteed to be convergent.
  \end{block}
\end{frame}
\begin{frame}
  \frametitle{Removing the Limitation on the Characteristic Equation}
  \begin{problem}[1*]
    Using a similar approach, find the general formula 
    of any homogeneous linear recursive requence where 
    its characteristic equation \emph{may} has 
    duplicate roots. 
  \end{problem}
\end{frame}
\begin{frame}
  \frametitle{Extensions}
\begin{problem}[2**]
  Prove for any recursive
  sequence with the recursive formula \[
    a_n = \sum_{r=1}^t m_r a_{n-r} + P(n)
  \] where $P(n)$ is a polynomial on $n$ and \[
    \sum_{r=1}^t m_r \neq 1
  \]
  there exists a calculabe general formula. \\
  Hint: start with $P(n) = c$ where $c$ is a constant.
\end{problem}
\end{frame}
\begin{frame}
  \frametitle{Extensions}
  \begin{problem}[3**]
     By setting column vector \[
       \textbf{V}_n = (a_{n-t}, a_{n-t+1}\dots a_n)^T
     \]
     and considering rewriting the recursive relationship
     as \[
       \textbf{V}_{n+1} = \textbf{R}\textbf{V}_n
     \] where $\textbf{R}$ is a $t \times t$
     matrix, reach the same result as the case 
     when all roots of the characteristic equation 
     are distinct.\\ Then, explain why its characteristic
     equation is defined in that way.
  \end{problem}
\end{frame}
\begin{frame}
  \huge{Thank You.}
\end{frame}
\end{document}
