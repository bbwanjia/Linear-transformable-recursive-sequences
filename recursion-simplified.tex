\documentclass{beamer}
\usetheme{Boadilla}
\usepackage{amsmath}
\usepackage{amssymb}
\theoremstyle{definition}
\newtheorem*{acknowledgement}{Acknowledgement}
\theoremstyle{remark}
\newtheorem*{remark}{Remark}
\title{Solving Recursive Sequences}
\subtitle{A Simple Glance}
\author{Liu Jingyu}
\date{\today}
\AtBeginSection[]{
  \begin{frame}
    \frametitle{Table of Contents}
    \tableofcontents[currentsection]
  \end{frame}
  }


\begin{document}
\frame{\titlepage}
\section{Introduction}
\begin{frame}
  \frametitle{What is a recursive sequence?}
  A recursive sequence is a sequence that its 
  terms are defined by preceding terms. Several 
  initial values $a_0, a_1 \dots$ are given and 
  all following terms are defined. 
  \begin{examples}[Recursive formyura]
     \[
       a_n = a_{n-1}
     \]
     \[
       a_n = 3a_{n-1} + 2
     \]
     \[
       a_n = a_{n-1} + a_{n-2}
     \]
  \end{examples}
Also for simplicity, we will only discuss linear
formulae.
\end{frame}
\section{Case of One Preceding Term}
\begin{frame}
  \frametitle{Case of One Preceding Term}
We will discuss sequences with the following 
recursice formula: \[
  a_n = m a_{n-1} + c, m \neq 0
\]
In our discussion, we will only focus on the 
recursive formulae, and not the initial values. 
\\
If from the recursive formulae can we find a general 
form of the general formula with some constants 
  unknown, then the sequence is said to be solved
  because those constants can be easily found by 
  substituting the initial values and solving 
  simultaneous equations. 

\end{frame}
\begin{frame}
  \frametitle{Trivial Cases}
For the recursive formula  $a_n = m a_{n-1} + c$\dots
  \begin{block}{when $m = 1$}
   It becomes an arithmetic sequence. 
   Its general formula is \[
    a_n = a_0 + nc.
   \]
\end{block}
  \begin{block}{when $c = 0$}
  It becomes a geometric sequence. 
    Its general formula is \[
      a_n =  m^n a_0
    \]
\end{block}
\end{frame}
\begin{frame}
  \frametitle{For non-trivial case $a_n = ma_{n-1} + c$}
  \begin{block}{for $m\neq 0, 1 \text{ and } c \neq 0$}
     If we could find some equivalent relationship like \[
       a_n - k = m(a_{n-1} - k) \]
     then we can treat $\{a_n - k\}$ as a geometric sequence. 
     \\
     From it
     we get \[
       a_n = ma_{n-1}+(1-m) k 
     \]
     so we let \(
       k = \frac{c}{1 - m}.
     \)
     By solving the geometric sequence 
     $\{a_n - k\}$ we get \[
       a_n - k = m^n (a_0 - k)
     \] and thus \[
      a_n = m^n (a_0 - \frac{c}{1 - m}) + \frac{c}{1 - m}.
     \]

   \end{block}
\end{frame}
\section{Case of Two Preceding Terms}
\begin{frame}
  \frametitle{Case of Two Preceding Terms}
  The linear recursive formula involving two preceding
  terms can be written as \[
    a_n = m_1 a_{n-1} + m_2 a_{n-2} + c
  \]
  Here we will discuss the situation where $c = 0$. 
  Similar methods can be used to solve the situation
  where $c \neq 0$ and $m_1 + m_2 \neq 1$. 
  \\
  Let's start by a famous example.
\end{frame}
\begin{frame}
  \frametitle{Fibanacci Sequence}
  The Fibonacci sequence is defined as $a_0 = 0$, $a_1 = 1$ and with 
  recursive formula $a_n = a_{n-1} + a_{n-2}$ for all $n=2,3,4\dots$ 
  \\
  It is well known that, for the Fibonacci sequence there 
  exists a closed-form general formula as below. 
  \\
  \begin{block}{Binet's Formula}
     The general formula of Fibonacci sequence is
     \[
       a_n = \frac{1}{\sqrt{5}} \left[\left(\frac{1+\sqrt{5}}{2}\right)^n
       -\left(\frac{1-\sqrt{5}}{2}\right)^n\right].
     \]
  \end{block}
\end{frame}
\begin{frame}
  \frametitle{Solving $a_n = m_1 a_{n-1} + m_2 a_{n-2}$}
  \begin{block}{Solution}
    Consider the recursive formula $a_n = m_1 a_{n-1} + m_2 a_{n-2}$. We want 
    to have \[
      a_n - \alpha a_{n-1} = \beta (a_{n-1} - \alpha a_{n-2})
    \]
    so that we can regard $\{a_n - \alpha a_{n-1}\}$ as a 
    geometric sequence. 
    By comparing the coefficients,
    \[
      \begin{cases}
         \alpha + \beta = m_1 \\
         -\alpha\beta = m_2
      \end{cases}
    \]
    So, $\alpha$ and $\beta$ are the roots of the equation \[
      x^2 - m_1 x -m_2 = 0. 
    \] 
  \end{block}
\end{frame}
\begin{frame}
  \begin{block}{Continued}
    \emph{Assuming the roots are distinct}, we notice that 
    the roots of the equation $x^2 - m_1 x -m_2 = 0$ namely $\alpha$ 
    and $\beta$ are symmetric, so we can write two equations:
    \[
      \begin{cases}
        a_n - \alpha a_{n-1} = \beta (a_{n-1} - \alpha a_{n-2}) \\
        a_n - \beta a_{n-1} = \alpha (a_{n-1} - \beta a_{n-2})
      \end{cases}
    \]
    By considering the initial values of these two 
    corresponding geometric sequences constants $A_1$ and $A_2$, applying
    the general formula for geometric sequences gives \[
      \begin{cases}
        a_n - \alpha a_{n-1} = A_1 \beta^n \\
        a_n - \beta a_{n-1} = A_2 \alpha^n 
      \end{cases}
    \]
    Simplifying give our final formula \[
      a_n = B_1 \alpha^n + B_2 \beta^n
    \]
    where $B_1$ and $B_2$ are constants to be found.
  \end{block}
\end{frame}
\begin{frame}
  \frametitle{An Example}
  \begin{examples}[Fibonacci]
    From the recursive formula $a_n = a_{n-1} + a_{n-2}$ we first 
    form an equation \[
      x^2 - x - 1 = 0.
    \]
    The roots of this equation are $\frac{1 \pm \sqrt{5}}{2}$, 
    so the general formula must be in the form \[
      a_n = A_1 \left(\frac{1+\sqrt{5}}{2}\right)^n
      + A_2 \left(\frac{1-\sqrt{5}}{2}\right)^n.
    \]
    Substituting $a_0 = 0$ and $a_1 = 1$ and solving 
    simultaneous equations directly give 
    \emph{Binet's Formula}.
   \end{examples}
\end{frame}
\section{A Note on Induction}
\begin{frame}
  \frametitle{A Note on Induction}
   All general formulae constructed above can be "found" 
   by mathematical induction if you can \emph{guess} the 
   answer from nothing. \\
   An example using induction is included in the 
   exercise. 
\end{frame}
\section{Further Thinking}
\begin{frame} 
  \begin{block}{Verify the following conjecture by induction}
     For a recursive sequence with the recursive formula \[
      a_n = \sum_{r=1}^t m_r a_{n-r},
     \]
     we have the general formula \[
       a_n = \sum_{r=1}^{t} C_r \lambda_r^n 
     \]
     where $C_1\dots C_t$ are constants and $\lambda_1\dots \lambda_r$
     are distinct roots of the equation \[
       x^t - \sum_{r=1}^t m_r x^{t-r} = 0.
     \] (All roots are distinct assumed)
  \end{block}
\end{frame}
\begin{frame}
  \frametitle{Extension to Matrices}
  \begin{problem}[3**]
    Let a sequence with recursive formula \[
      a_n = m_1 a_{n-1} + m_2 a_{n-2}
    \] 
    and $m_1^2 + 4m_2 > 0$. \\
    By setting column vector \[
       \textbf{V}_n = \begin{pmatrix}
         a_{n-1} \\ a_n 
       \end{pmatrix}
     \]
     and considering rewriting the recursive relationship
     as \[
       \textbf{V}_{n+1} = \textbf{R}\textbf{V}_n
     \] where $\textbf{R}$ is a $2 \times 2$
     matrix, reach the same result. 
\end{problem}
\end{frame}
\begin{frame}
  \huge{Thank You.}
\end{frame}
\end{document}
